\documentclass{article}
\usepackage{graphicx} % Required for inserting images
\usepackage{listings} % Required for displaying code nicely
\usepackage{xcolor}
\usepackage[margin=1in]{geometry} %forcing 1inch margins 
\usepackage{tabularx}
\usepackage{float}
\usepackage{minted}
\sloppy


\renewcommand{\familydefault}{\ttdefault}


\definecolor{codegreen}{HTML}{59636F} % comments
\definecolor{codegray}{HTML}{1E2229} % numbers?
\definecolor{codepurple}{HTML}{0A3068} %strings
%\definecolor{backcolour}{rgb}{0.95,0.95,0.92} %bg
\definecolor{backcolour}{HTML}{F2F3F4}
\definecolor{keycolour}{HTML}{CE232E}


\lstdefinestyle{codeStyle}{
    backgroundcolor=\color{backcolour},   
    commentstyle=\color{codegreen},
    keywordstyle=\color{keycolour},
    numberstyle=\tiny\color{codegray},
    stringstyle=\color{codepurple},
    basicstyle=\ttfamily\footnotesize,
    breakatwhitespace=false,         
    breaklines=true,                 
    captionpos=b,                    
    keepspaces=true,                 
    numbers=left,                    
    numbersep=5pt,                  
    showspaces=false,                
    showstringspaces=false,
    showtabs=false,                  
    tabsize=2
}

\lstset{style=codeStyle}


\title{COMP1314 Coursework}
\author{Oran Keating (ok1g24@soton.ac.uk)}
\date{December 2024}

\begin{document}

\maketitle

\section{Structured Data}

\subsection{Ex1}

The bash script below converts both students.xml and faculty.xml into CSV formats. Its first two "if" statements are to ensure the correct number of command line arguments are given and that, when give, they are the correct arguments. The final two "if" statements actually convert the files into CSV's. The code used to convert the scripts is by no means the most efficient way of conversion nor is it able to be used for any xml script hence the two different versions for facutly.xml and student.xml. I will explain what each part of the command to actually convert the files does below the script. 

\begin{lstlisting}[language=Bash, caption=XML to CSV script]
#!/bin/bash

#Ensuring correct No. of command line arguments are given
if [[ $# != 1 ]]; then
	echo "Error: one command Line argument must be provided"
	exit 1
fi


if [[ "$1" != "faculty.xml" && "$1" != "students.xml" ]]; then
	echo "Error: command line arguemt must be either faculty.xml or students.xml"
	exit 1
fi

#Processing data and outputting to CSV
if [[ "$1" == "faculty.xml" ]]; then
	echo "Writing to faculty.csv..."
	echo "faculty,building,room,capacity" > faculty.csv
	grep -v xml "$1" | sed -E 's/<\/?[a-z]*>//g' | awk '{$1=$1; print}' | tr -d '\r' | sed '/./ N; s/\n/,/g' | sed '/./ N; s/\n/,/g' | awk 'NF {printf"%s\n",$0}' >> faculty.csv
fi

if [[ "$1" == "students.xml" ]]; then
	echo "Writing to students.csv..."
	echo "student_name,student_id,student_email,programme,year,address,contact,module_id,module_name,module_leader,lecturer1,lecturer2,faculty,building,room,exam_mark,coursework1,coursework2,coursework3" > students.csv
	grep -v xml "$1" | sed 's|<[^>]*\/>|<1>NULL<\/1>|g' | awk '{$1=$1;print}'| sed -e 's|\(^[^< ]\)| \1|g' | tr -d '\r\n' | sed -e 's|\([^>]*,[^<]*\)|"\1"|g' | sed 's|<\/student>|\n|g' | sed 's|<[^\/]*>||g' | sed 's|<\/[^>]*>|,|g' | sed 's|,$||g' >> students.csv
	
fi

echo "Done"

\end{lstlisting}


\begin{table}[H]
    \centering
    \begin{tabularx}{\textwidth}{|X|X|} % 3 columns that auto-adjust
        \hline
        \textbf{Code} & \textbf{Explanation} \\
        \hline
        \verb|grep -v xml "$1"| & Removing XML version tag \\
        \hline
        \verb|sed -E 's/<\/?[a-z]*>//g'| & \verb|Removing <> and </> tags| \\
        \hline 
        \verb|awk '{$1=$1; print}'| & Removing leading and trailing whitespace \\
        \hline 
        \verb|tr -d '\r'| & \texttt{Removes \verb|\r| as file has been opened in both Linux and windows} \\
        \hline 
        \verb|sed '/./ N; s/\n/,/g'| & Combines 2 lines into one by replacing newlines with commas. Only does this for the non-empty lines as specified by the /./ and takes into account the next line too due to the "N;" \\
        \hline 
        \verb|sed '/./ N; s/\n/,/g'| & Same as above as we need to combine a total of 4 lines into one \\
        \hline 
        \texttt{\verb|awk 'NF \{printf"\%s\\n",\$0\}'|} & Removes 2 blank lines in between each line of text by printing each line verbatim with a \verb|\n| at the end to separate them \\
        \hline
        \texttt{> faculty.csv} & Piping output into file called faculty.csv \\
        \hline
    \end{tabularx}
    \caption{Explanation of code for faculty.xml}
    %\label{tab:sample_table}
\end{table}

\begin{table}[H]
    \centering
    \begin{tabularx}{\textwidth}{|X|X|}
    \hline
    \multicolumn{2}{|X|}{Note: sed commands are not fully shown due to some characters being delimiters in LaTeX verbatim environments. However, they are in the correct order.} \\
        \hline
        \textbf{Code} & \textbf{Explanation} \\
        \hline
        \verb|grep -v xml "$1"| & Removing XML version tag \\
        \hline
        sed ... & Replacing empty xml tags with <1>NULL<\1> to be able to identify them later \\
        \hline
        \verb|awk '{ $1 = $1 ; print } '| & Removing sorrounding whitespace \\
        \hline
        sed ... & Adds a space to any line that does not start with "<". This means that when later formatting the address line there remains a space when two lines are merged into one. \\
        \hline
        \verb|tr -d '\ r \ n '| & Removes carriage returns and newlines \\
        \hline
        sed -e ... & Puts "" around text inside tags that contains a comma. This is so the comma can remain in the CSV file without it being parsed as a delimiter. \\
        \hline
        sed ...student... & Replaces closing student tag with a newline to separate each student. Allows each student to have their own row when it eventually becomes a CSV file.  \\
        \hline
        sed ... & Removes all opening XML tags. \\
        \hline
        sed ... & Replaces all closing XML tags with a comma. \\
        \hline
        sed 's\textbar,\$\textbar\textbar g' & Removes all commas at the end of a line to properly seperate each student \\
        \hline
        \texttt{>> students.csv} & Specifically appending to students.csv as the header line is written previously \\
        \hline
    \end{tabularx}
    \caption{Explanation of code for students.xml}
\end{table}

\newpage

\subsection{Ex2}

\begin{lstlisting}[language=Bash, caption=CSV to TXT script]
#!/bin/bash

if [[ "$#" -ne 2 ]]; then
        echo "Error: Two command line arguments must be provided"
        exit 1
fi


if [[ "$1" != "students.csv" ]]; then
        echo "Error: First command line argument must be students.csv"
        exit 1
fi

if [[ "$#" == 2 && "$2" != *.txt ]]; then
	echo "Error: The second argument must be a text file"
	exit 1
fi

file_name="$2"

echo "Writing to $file_name..."

cut -d, -f1,2 $1 | tail -n +2 | sort -t, -k1,1 | uniq -f1 | cut -d, -f1 > $file_name

echo "Done"
\end{lstlisting}

\begin{table}[H]
    \centering
    \begin{tabularx}{\textwidth}{|X|X|} % 3 columns that auto-adjust
        \hline
        \textbf{Code} & \textbf{Explanation} \\
        \hline
        \verb|cut -d, -f1,2 $1| & Splits the file given by the first command line argument by commas and takes the first and second row \\
        \hline
        \verb|tail -n +2|& Removes the first line from the file as the file contains the row headers \\
        \hline
        \verb|sort -t, -k1,1| & Sorts by first column using a comma as the field separator  \\
        \hline
        \verb|uniq -f1| & deletes all duplicate student\_ids as it skips the first field \\
        \hline
        \verb|cut -d, -f1| & Splits by comma and takes the first row \\
        \hline
        \verb|> $file_name| & writes output to file \\
        \hline
    \end{tabularx}
    \caption{Explanation of code for generating a list of names from a csv}
    %\label{tab:sample_table}
\end{table}

\section{Relational Model}
\subsection{Ex3}
{\textbf {Students relation:}}\\ 
\texttt{Students(student\_name, student\_id, student\_email, programme, year, address, contact\_number, module\_id, module\_name, module\_leader, lecturer1, lecturer2, faculty, building, room, exam\_mark, coursework1, coursework2, coursework3)\\\\}
Faculty relation:\\ 
\verb|Faculty(faculty,building,room,capacity)|

\newpage
\subsection{Ex4}
{\Large Assumptions}
\newline The following are assumptions that have been made about the wider domain of the dataset:

\begin{itemize}
    \item Students can not take more than one program
    \item Students do not have more than one contact number
    \item The uni will never re-use email addresses
    \item More than one faculty can use the same room in the same building
    \item A module will only ever use the same room.
    \item A module will only ever have the same lead and lecturers
    \item A program is not run by more than one faculty
    \item Only one faculty can run a module but multiple programs can take that module
    \item Lecturer 1 and Lecturer 2 are not two distinct attributes and their order can be swapped 
    \item Coursework1, 2 and 3 are three distinct attributes who's order matters and can not be swapped
\end{itemize}

\noindent
{\Large Minimal set of functional dependencies}\\
Students Relation:\\
\noindent
\verb|student_id -> student_name|\newline
\verb|student_id -> student_email|\newline
\verb|student_id -> programme|\newline
\verb|student_id -> year|\newline
\verb|student_id -> address|\newline
\verb|student_id <-> contact|\newline
\verb|student_email <-> contact|\\


\noindent
\verb|student_id, module_id -> exam_mark|\\
\verb|student_id, module_id -> coursework1|\\
\verb|student_id, module_id -> coursework2|\\
\verb|student_id, module_id -> coursework3|\\

\noindent
\verb|module_id -> module_name|\\
\verb|module_id -> module_leader|\\
\verb|module_id -> lecturer1|\\
\verb|module_id -> lecturer2|\\
\verb|module_id -> faculty|\\
\verb|module_id -> building|\\
\verb|module_id -> room|\\

\noindent
\verb|programme -> faculty|\\


\noindent
Faculty Relation:\\
\verb|building, room -> capacity|

\subsection{Ex5}
Faculty relation candidate keys:
\begin{itemize}
    \item Composite Primary Key: (faculty,building,room)
\end{itemize}

\noindent
Students relation candidate keys:
\begin{itemize}
    \item Composite Primary Key:
    (student\_id,module\_id)
    \item Composite Primary Key:
    (student\_email,module\_id)
    \item Composite Primary Key:
    (contact\_number,module\_id)
\end{itemize}

\subsection{Ex6}
{\large Faculty relation:}\\
Primary composite key: (faculty,building,room)\\
This is because this is the only candidate key\\\\
{\large Students relation:}\\
Primary composite key: (student\_id, module\_id)\\
This is because an id number is a common and known way to uniquely identify people in a large organization and deviating from the norm is likely to cause confusion. In addition students may change their email or contact number over time and this will result in have to update the database however they will never change their id. 


\section{Normalisation}
\subsection{Ex7}
The data is not in first normal from. There are repeating groups such as (lecturer1, lecturer2) and (coursework1,coursework2,coursework3)\\

\noindent
The set of relations is as follows:
\begin{itemize}
    \item Students(student\_name, student\_id, student\_email, programme, year, address, contact\_number, module\_id, module\_name, module\_leader, faculty, building, room, exam\_mark)
    \item Lecturers(module\_id, lecturer)
    \item Courseworks(student\_id, module\_id, coursework\_number, coursework\_mark)
    \item Faculty(faculty,building,room,capacity)
\end{itemize}

\subsection{Ex8}
In the existing dataset there is missing data and this has been replaced with NULL data. As the lecturer number was not of importance it was able to be decomposed into a table with just module\_id and lecturer however because coursework1, 2, 3 detonated which coursework the mark was for a coursework\_number column is required.

\subsection{Ex9}


Partial dependency's in faculty relation:
\begin{itemize}
    \item building, room -> capacity
\end{itemize}

\noindent
New relations to be made as a result:
\begin{itemize}
    \item Room\_Capacity(building,room,capacity)
    \item Faculty\_Room(faculty,building,room)
\end{itemize}


\noindent
\\
Partial dependency's in students relation:

\begin{itemize}
    \item \verb|module_id -> module_name|
    \item \verb|module_id -> module_leader|
    \item \verb|module_id -> faculty|
    \item \verb|module_id -> building|
    \item \verb|module_id -> room|
    \item \verb|student_id -> student_name|
    \item \verb|student_id -> student_email|
    \item \verb|student_id -> programme|
    \item \verb|student_id -> year|
    \item \verb|student_id -> address|
    \item \verb|student_id -> contact|
\end{itemize}

\noindent
New relations to be made as a result:
\begin{itemize}
    \item student\_info(student\_name, student\_id, student\_email, programme, year, address, contact)
    \item students\_modules(student\_id, module\_id, exam\_mark)
    \item Modules(module\_id, module\_name, module\_leader, faculty, building, room)
\end{itemize}


\subsection{Ex10}

Decomposing faculty:
In the faculty relation there is one partial dependency. Capacity is only dependent on building and room but not faculty. To resolve this the following relations can be created and the existing faculty relation removed
\begin{itemize}
    \item Room\_Capacity(building,room,capacity)
    \item Faculty\_Room(faculty,building,room)
\end{itemize}

\noindent
Decomposing students:
Students can first be broken up into the following two relations:
\begin{itemize}
    \item Modules(module\_id, module\_name, module\_leader, faculty, building, room)
    \item Students(student\_name, student\_id, student\_email, programme, year, address, contact, module\_id, exam\_mark
\end{itemize}

However in the second of these two relations there still exists partial dependencies. As the key is a composite key consisting of (student\_id, module\_id) all attributes other than exam mark rely on solely student\_id. We therefore have to further decompose this relation into two new ones.



\begin{itemize}
    \item student\_info(student\_name, student\_id, student\_email, programme, year, address, contact)
    \item students\_modules(student\_id, module\_id, exam\_mark)
\end{itemize}

\par\noindent\rule{\textwidth}{1pt}\\

\centering{\large List of current relations, fields and types}


\begin{table}[H]
    \centering
    \begin{tabularx}{\textwidth}{|X|X|} % 3 columns that auto-adjust
    \hline
    \multicolumn{2}{|c|}{\textbf{room\_capacity relation}} \\
        \hline
    \multicolumn{2}{|c|}{\textbf{Primary Key: building, room}} \\
    \hline
        \textbf{Field} & \textbf{Type} \\
        \hline
        \verb|building| & str \\
        \hline
        \verb|room| & str \\
        \hline
        \verb|capacity| & int \\
        \hline
    \end{tabularx}
\end{table}

\begin{table}[H]
    \centering
    \begin{tabularx}{\textwidth}{|X|X|} % 3 columns that auto-adjust
    \hline
    \multicolumn{2}{|c|}{\textbf{faculties\_room relation}} \\
        \hline
    \multicolumn{2}{|c|}{\textbf{Primary Key: building, room}} \\
    \hline
    \multicolumn{2}{|c|}{\textbf{Foreign Keys: (building, room) -> room\_capacity}} \\
    \hline
        \textbf{Field} & \textbf{Type} \\
        \hline
        \verb|faculty| & str \\
        \hline
        \verb|building| & str \\
        \hline
        \verb|room| & str \\
        \hline
    \end{tabularx}
\end{table}

\begin{table}[H]
    \centering
    \begin{tabularx}{\textwidth}{|X|X|}
    \hline
    \multicolumn{2}{|c|}{\textbf{student\_info relation}} \\
    \hline
        \multicolumn{2}{|c|}{\textbf{Primary Key: student\_id}} \\
        \hline
        \textbf{Field} & \textbf{Type} \\
        \hline
        \verb|student_name| & str \\
        \hline
        \verb|student_id| & str \\
        \hline
        \verb|student_email| & str \\
        \hline
        \verb|programme| & str \\
        \hline
        \verb|year| & int \\
        \hline
        \verb|address| & str \\
        \hline
        \verb|contact| & str \\
        \hline
    \end{tabularx}
\end{table}

\begin{table}[H]
    \centering
    \begin{tabularx}{\textwidth}{|X|X|} % 3 columns that auto-adjust
    \hline
    \multicolumn{2}{|c|}{\textbf{modules relation}} \\
        \hline
    \multicolumn{2}{|c|}{\textbf{Primary Key: module\_id}} \\
    \hline
    \multicolumn{2}{|c|}{\textbf{Foreign Keys: (building, room) -> room\_capacity}} \\
    \hline
        \textbf{Field} & \textbf{Type} \\
        \hline
        \verb|module_id| & str \\
        \hline
        \verb|module_name| & str \\
        \hline
        \verb|module_leader| & str \\
        \hline
        \verb|faculty| & str \\
        \hline
        \verb|building| & str \\
        \hline
        \verb|room| & str \\
        \hline
    \end{tabularx}
\end{table}

\begin{table}[H]
    \centering
    \begin{tabularx}{\textwidth}{|X|X|} 
    \hline
    \multicolumn{2}{|c|}{\textbf{student\_modules relation}} \\
        \hline
    \multicolumn{2}{|c|}{\textbf{Primary Key: student\_id, module\_id}} \\
    \hline
    \multicolumn{2}{|c|}{\textbf{Foreign Keys: student\_id -> student\_info,  module\_id -> modules}} \\
    \hline
        \textbf{Field} & \textbf{Type} \\
        \hline
        \verb|student_id| & str \\
        \hline
        \verb|module_id| & str \\
        \hline
        \verb|exam_mark| & int \\
        \hline
    \end{tabularx}
\end{table}



\begin{table}[H]
    \centering
    \begin{tabularx}{\textwidth}{|X|X|} % 3 columns that auto-adjust
    \hline
    \multicolumn{2}{|c|}{\textbf{lecturers relation}} \\
        \hline
    \multicolumn{2}{|c|}{\textbf{Primary Key: module\_id, lecturer}} \\
    \hline
    \multicolumn{2}{|c|}{\textbf{Foreign Keys: module\_id -> modules}} \\
    \hline
        \textbf{Field} & \textbf{Type} \\
        \hline
        \verb|module_id| & str \\
        \hline
        \verb|lecturer| & str \\
        \hline
    \end{tabularx}
\end{table}

\begin{table}[H]
    \centering
    \begin{tabularx}{\textwidth}{|X|X|} % 3 columns that auto-adjust
    \hline
    \multicolumn{2}{|c|}{\textbf{courseworks relation}} \\
        \hline
    \multicolumn{2}{|c|}{\textbf{Primary Key: student\_id, module\_id, coursework\_number}} \\
    \hline
    \multicolumn{2}{|c|}{\textbf{Foreign Keys: student\_id -> student\_info,  module\_id -> modules}} \\
    \hline
        \textbf{Field} & \textbf{Type} \\
        \hline
        \verb|student_id| & str \\
        \hline
        \verb|module_id| & str \\
        \hline
        \verb|coursework_number| & int \\
        \hline
        \verb|coursework_mark| & int \\
        \hline
    \end{tabularx}
\end{table}


\raggedright
\subsection{Ex11}
For the faculty relation the primary key was a composite key consisting of building,room,faculty however a partial dependency existed as capacity only relied on building,room and not faculty. Decomposing it into two tables both with the primary keys of building,room ensures the table is in 2NF.\newline

For the student relation two separate decompositions had to be made before the relations were in 2NF. One to split students and modules and another to split the new students relation into student\_info and student\_modules.\newline

When decomposing in both case I removed the minimum amount of attributes possible to a new table in order to keep the tables as simple as possible and not risk violating 2NF. 

\subsection{Ex12}
There is one translative dependency in the modules relation
\begin{itemize}
    \item module\_id -> building, room -> faculty
\end{itemize}


\newpage
\subsection{Ex13}
To fix the transitive dependency the "faculty" atribute is removed from the "modules" relation. No new relations need to be created as there already exists a relation linking "faculty" with "building" and "room", that being the "faculties\_rooms" relation. Below is the adjusted relation.


\begin{table}[H]
    \centering
    \begin{tabularx}{\textwidth}{|X|X|} % 3 columns that auto-adjust
    \hline
    \multicolumn{2}{|c|}{\textbf{modules Relation}} \\
        \hline
    \multicolumn{2}{|c|}{\textbf{Primary Key: student\_id, module\_id}} \\
    \hline
        \textbf{Field} & \textbf{Type} \\
        \hline
        \verb|module_id| & str \\
        \hline
        \verb|module_name| & str \\
        \hline
        \verb|module_leader| & str \\
        \hline
        \verb|building| & str \\
        \hline
        \verb|room| & str \\
        \hline
    \end{tabularx}
\end{table}


\section{Modelling}
\subsection{Ex14}

\begin{table}[H]
    \centering
    \begin{tabularx}{\textwidth}{|X|X|}
    \hline
    \multicolumn{2}{|c|}{\textbf{student\_info Relation}} \\
    \hline
        \multicolumn{2}{|c|}{\textbf{Primary Key: student\_id}} \\
        \hline
        \textbf{Field} & \textbf{SQLite Datatype} \\
        \hline
        \verb|student_name| & TEXT \\
        \hline
        \verb|student_id| & INTEGER \\
        \hline
        \verb|student_email| & TEXT \\
        \hline
        \verb|programme| & TEXT \\
        \hline
        \verb|year| & INTEGER \\
        \hline
        \verb|address| & TEXT \\
        \hline
        \verb|contact| & TEXT \\
        \hline
    \end{tabularx}
\end{table}

\begin{table}[H]
    \centering
    \begin{tabularx}{\textwidth}{|X|X|} 
    \hline
    \multicolumn{2}{|c|}{\textbf{student\_modules Relation}} \\
        \hline
    \multicolumn{2}{|c|}{\textbf{Primary Key: student\_id, module\_id}} \\
    \hline
        \textbf{Field} & \textbf{SQLite Datatype} \\
        \hline
        \verb|student_id| & INTEGER \\
        \hline
        \verb|module_id| & TEXT \\
        \hline
        \verb|exam_mark| & INTEGER \\
        \hline
    \end{tabularx}
\end{table}

\begin{table}[H]
    \centering
    \begin{tabularx}{\textwidth}{|X|X|} % 3 columns that auto-adjust
    \hline
    \multicolumn{2}{|c|}{\textbf{modules Relation}} \\
        \hline
    \multicolumn{2}{|c|}{\textbf{Primary Key: module\_id}} \\
    \hline
        \textbf{Field} & \textbf{SQLite Datatype} \\
        \hline
        \verb|module_id| & TEXT \\
        \hline
        \verb|module_name| & TEXT \\
        \hline
        \verb|module_leader| & TEXT \\
        \hline
        \verb|building| & TEXT \\
        \hline
        \verb|room| & TEXT \\
        \hline
    \end{tabularx}
\end{table}

\begin{table}[H]
    \centering
    \begin{tabularx}{\textwidth}{|X|X|} % 3 columns that auto-adjust
    \hline
    \multicolumn{2}{|c|}{\textbf{lecturers Relation}} \\
        \hline
    \multicolumn{2}{|c|}{\textbf{Primary Key: module\_id, lecturer}} \\
    \hline
        \textbf{Field} & \textbf{SQLite Datatype} \\
        \hline
        \verb|module_id| & TEXT \\
        \hline
        \verb|lecturer| & TEXT \\
        \hline
    \end{tabularx}
\end{table}

\begin{table}[H]
    \centering
    \begin{tabularx}{\textwidth}{|X|X|} % 3 columns that auto-adjust
    \hline
    \multicolumn{2}{|c|}{\textbf{courseworks Relation}} \\
        \hline
    \multicolumn{2}{|c|}{\textbf{Primary Key: student\_id, module\_id, coursework\_number}} \\
    \hline
        \textbf{Field} & \textbf{SQLite Datatype} \\
        \hline
        \verb|student_id| & INTEGER \\
        \hline
        \verb|module_id| & TEXT \\
        \hline
        \verb|coursework_number| & INTEGER \\
        \hline
        \verb|coursework_mark| & INTEGER \\
        \hline
    \end{tabularx}
\end{table}

\begin{table}[H]
    \centering
    \begin{tabularx}{\textwidth}{|X|X|} % 3 columns that auto-adjust
    \hline
    \multicolumn{2}{|c|}{\textbf{room\_capacity Relation}} \\
        \hline
    \multicolumn{2}{|c|}{\textbf{Primary Key: building, room}} \\
    \hline
        \textbf{Field} & \textbf{SQLite Datatype} \\
        \hline
        \verb|building| & TEXT \\
        \hline
        \verb|room| & TEXT \\
        \hline
        \verb|capacity| & TEXT \\
        \hline
    \end{tabularx}
\end{table}

\begin{table}[H]
    \centering
    \begin{tabularx}{\textwidth}{|X|X|} % 3 columns that auto-adjust
    \hline
    \multicolumn{2}{|c|}{\textbf{faculties\_room Relation}} \\
        \hline
    \multicolumn{2}{|c|}{\textbf{Primary Key: building, room}} \\
    \hline
        \textbf{Field} & \textbf{SQLite Datatype} \\
        \hline
        \verb|faculty| & TEXT \\
        \hline
        \verb|building| & TEXT \\
        \hline
        \verb|room| & TEXT \\
        \hline
    \end{tabularx}
\end{table}






\subsection{Ex15}

\begin{lstlisting}[language=SQLite3, caption=SQL Code to import students.csv and faculty.csv]
.import students.csv studentscsv
.import faculty.csv facultycsv
.output ex15.sql
.dump
\end{lstlisting}


\subsection{Ex16}

The NULLIF SQL command will convert any "NULL" strings it finds to actual null values in the tables.

\begin{lstlisting}[language=SQL, caption=SQL Code to create and populate the tables in the database]
PRAGMA foreign_keys = ON;

--FACULTY RELATIONS

CREATE TABLE IF NOT EXISTS room_capacity (
    building TEXT NOT NULL,
    room TEXT NOT NULL,
    capacity INTEGER DEFAULT 0 CHECK (capacity >= 0),
    PRIMARY KEY (building, room)
    );

CREATE TABLE IF NOT EXISTS faculty_room (
    building TEXT NOT NULL,
    room TEXT NOT NULL,
    faculty TEXT NOT NULL,
    PRIMARY KEY (building, room)
    );


INSERT INTO room_capacity (building, room, capacity)
SELECT DISTINCT building, room, capacity
FROM facultycsv;

INSERT INTO faculty_room (building, room, faculty)
SELECT DISTINCT building, room, faculty
FROM facultycsv;


--STUDENT RELATIONS

CREATE TABLE IF NOT EXISTS student_info (
student_name TEXT, 
student_id INTEGER NOT NULL CHECK (student_id >= 0), 
student_email TEXT, 
programme TEXT, 
year INTEGER CHECK (year > 0), 
address TEXT, 
contact TEXT,
PRIMARY KEY (student_id)
);

INSERT INTO student_info (student_name, student_id, student_email, programme, year, address, contact)
SELECT DISTINCT NULLIF(student_name, 'NULL'), 
                student_id, 
                NULLIF(student_email, 'NULL'), 
                NULLIF(programme, 'NULL'), 
                NULLIF(year, 'NULL'),
                NULLIF(address, 'NULL'),
                NULLIF(contact, 'NULL')
FROM studentscsv;

CREATE TABLE IF NOT EXISTS modules (
module_id TEXT NOT NULL,
module_name TEXT NOT NULL,
module_leader TEXT,
building TEXT,
room TEXT,
PRIMARY KEY (module_id),
FOREIGN KEY (building, room) REFERENCES room_capacity(building, room)
);

INSERT INTO modules (module_id, module_name, module_leader, building, room)
SELECT DISTINCT module_id, module_name, NULLIF(module_leader, 'NULL'), NULLIF(building, 'NULL'), NULLIF(room, 'NULL')
FROM studentscsv;

CREATE TABLE IF NOT EXISTS student_modules (
student_id INTEGER NOT NULL,
module_id TEXT NOT NULL,
exam_mark INTEGER CHECK (exam_mark >= 0),
PRIMARY KEY (student_id, module_id),
FOREIGN KEY (student_id) REFERENCES student_info(student_id),
FOREIGN KEY (module_id) REFERENCES modules(module_id)
);

--ROWS should be inserted regardless of NULL exam_mark as student still takes the module
INSERT INTO student_modules (student_id, module_id, exam_mark)
SELECT DISTINCT student_id, module_id, NULLIF(exam_mark, 'NULL')
FROM studentscsv;


CREATE TABLE IF NOT EXISTS lecturers (
module_id TEXT NOT NULL,
lecturer TEXT,
PRIMARY KEY (module_id, lecturer)
FOREIGN KEY (module_id) REFERENCES modules(module_id)
);

INSERT INTO lecturers (module_id, lecturer)
SELECT DISTINCT module_id, lecturer1 AS lecturer
FROM studentscsv
WHERE lecturer1 != 'NULL';

INSERT INTO lecturers (module_id, lecturer)
SELECT DISTINCT module_id, lecturer2 AS lecturer
FROM studentscsv
WHERE lecturer2 != 'NULL';

CREATE TABLE IF NOT EXISTS courseworks (
student_id INTEGER NOT NULL,
module_id TEXT NOT NULL,
coursework_number INTEGER CHECK (coursework_number > 0),
coursework_mark INTEGER CHECK (coursework_mark >= 0),
PRIMARY KEY (student_id, module_id, coursework_number),
FOREIGN KEY (student_id) REFERENCES student_info(student_id),
FOREIGN KEY (module_id) REFERENCES modules(module_id)
);

INSERT INTO courseworks (student_id, module_id, coursework_number, coursework_mark)
SELECT DISTINCT student_id, module_id, 1 AS coursework_number, NULLIF(coursework1, 'NULL') AS coursework_mark
FROM studentscsv;

INSERT INTO courseworks (student_id, module_id, coursework_number, coursework_mark)
SELECT DISTINCT student_id, module_id, 2 AS coursework_number, NULLIF(coursework2, 'NULL') AS coursework_mark
FROM studentscsv;

INSERT INTO courseworks (student_id, module_id, coursework_number, coursework_mark)
SELECT DISTINCT student_id, module_id, 3 AS coursework_number, NULLIF(coursework3, 'NULL') AS coursework_mark
FROM studentscsv;

\end{lstlisting}

\section{Querying}

\subsection{Ex17}
\begin{lstlisting}[language=SQL]
SELECT building, SUM(capacity)
FROM room_capacity
GROUP BY building;
\end{lstlisting}
\subsection{Ex18}
\begin{lstlisting}[language=SQL]
SELECT student_info.student_id, student_info.student_name, AVG(student_modules.exam_mark) AS avg_exam_mark
FROM student_info
JOIN student_modules ON student_info.student_id = student_modules.student_id
WHERE exam_mark IS NOT NULL AND student_info.year = 1 AND student_info.programme = 'Computer Science'
GROUP BY student_info.student_id
ORDER BY avg_exam_mark DESC;
\end{lstlisting}
\subsection{Ex19}
\begin{lstlisting}[language=SQL]
SELECT module_id, module_leader, faculty, MAX(avg_marks)
FROM(
    SELECT m.module_id, m.module_leader, fr.faculty, 
        AVG(
            CASE 
                WHEN sm.exam_mark IS NOT NULL OR cw.coursework_mark IS NOT NULL 
                THEN sm.exam_mark + cw.coursework_mark 
                WHEN sm.exam_mark IS NOT NULL OR cw.coursework_mark IS NULL
                THEN sm.exam_mark
                WHEN sm.exam_mark IS  NULL OR cw.coursework_mark IS NOT NULL
                THEN cw.coursework_mark
            END
        ) AS avg_marks
    FROM modules m
    JOIN faculty_room fr ON (m.building = fr.building AND m.room = fr.room)
    JOIN student_modules sm ON m.module_id = sm.module_id
    JOIN courseworks cw ON (m.module_id = cw.module_id AND sm.student_id = cw.student_id)
    GROUP BY m.module_id, m.module_leader, fr.faculty
)
GROUP BY faculty;
\end{lstlisting}
\subsection{Ex20}
\begin{lstlisting}[language=SQL]
SELECT m.module_id, rc.building, rc.room
FROM modules m
JOIN room_capacity rc on (m.building = rc.building AND m.room = rc.room)
JOIN student_modules sm ON m.module_id = sm.module_id
GROUP BY m.module_id, rc.building, rc.room
HAVING COUNT(sm.student_id) > rc.capacity;
\end{lstlisting}



\end{document}
